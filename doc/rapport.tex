\documentclass[pdftex,10pt,a4paper]{article}
\usepackage[utf8]{inputenc}
\usepackage[francais]{babel}
\usepackage{hyperref}
\usepackage{eurosym}
\usepackage[a4paper]{geometry}
\usepackage{graphicx}

\geometry{hscale=0.75,vscale=0.80,centering}
\newcommand{\HRule}{\rule{\linewidth}{0.5mm}}


\begin{document}

\begin{titlepage}
\begin{center}

\includegraphics[width=0.45\textwidth]{./sceau-a-quadri.jpg}\\[1cm]

\textsc{\LARGE Université Libre de Bruxelles}\\[1.5cm]
\textsc{\Large INFO-F-403\\ Introduction to language theory and compiling}\\[1.5cm]

\HRule \\[0.4cm]
{ \huge \bfseries Assignement 1 : The scanner\\[0.4cm] }

\HRule \\[1.5cm]

\center{Maxime \textsc{Desclefs} - 362626}
\center{Julian \textsc{Schembri} - 380446}

\vfill
{\large 2015 - 2016}

\end{center}
\end{titlepage}

\newpage

\section{Introduction}

Dans le cadre du cours d'introduction à la théorie du langage et de compilation (INFO-F-403), il a été demandé d'implémenter et d'écrire un compilateur pour le langage SUPRALGOL$^{2016}$, une variante du langage Algol68.\\\\ La première partie consiste en la création du scanner. Ce scanner analyse et découpe le code en différents tokens différenciant chacun des éléments du langage.


\section{Tokens}

Voici le tableau qui reprend l'ensemble des tokens du langage Supralgol. Chaque ligne de celui-ci est composée du nom du token, le type dans l'enum LexicalUnit qui lui est associé et de l'expression régulière qui correspond à ce token.\\

\begin{tabular}{|l|c|r|}
  \hline
  Nom & Type & Expression régulière \\
  \hline
  Comment & - & \begin{minipage}{2in} \begin{verbatim} co.*co \end{verbatim} \end{minipage} \\
  LeftParenthesis (() & LEFT\_PARENTHESIS & \begin{minipage}{2in} \begin{verbatim} ( \end{verbatim} \end{minipage} \\
  RightParenthesis ()) & RIGHT\_PARENTHESIS & \begin{minipage}{2in} \begin{verbatim} ) \end{verbatim} \end{minipage} \\

  Beg & BEGIN & \begin{minipage}{2in} \begin{verbatim} begin \end{verbatim} \end{minipage} \\
  End & END & \begin{minipage}{2in} \begin{verbatim} end \end{verbatim} \end{minipage} \\

  Semicolon (;) & SEMICOLON & \begin{minipage}{2in} \begin{verbatim} ; \end{verbatim} \end{minipage} \\
  Assign (:=) & ASSING & \begin{minipage}{2in} \begin{verbatim} := \end{verbatim} \end{minipage} \\

  Minus (-) & MINUS & \begin{minipage}{2in} \begin{verbatim} - \end{verbatim} \end{minipage} \\
  Plus (+) & PLUS & \begin{minipage}{2in} \begin{verbatim} + \end{verbatim} \end{minipage} \\
  Times (*) & TIMES & \begin{minipage}{2in} \begin{verbatim} * \end{verbatim} \end{minipage} \\
  Divide (/) & DIVIDE & \begin{minipage}{2in} \begin{verbatim} / \end{verbatim} \end{minipage} \\


  If & IF & \begin{minipage}{2in} \begin{verbatim} if \end{verbatim} \end{minipage} \\
  Then & THEN & \begin{minipage}{2in} \begin{verbatim} then \end{verbatim} \end{minipage} \\
  Fi & FI & \begin{minipage}{2in} \begin{verbatim} fi \end{verbatim} \end{minipage} \\
  Else & ELSE & \begin{minipage}{2in} \begin{verbatim} else \end{verbatim} \end{minipage} \\
  Not & NOT & \begin{minipage}{2in} \begin{verbatim} not \end{verbatim} \end{minipage} \\

  Or & OR & \begin{minipage}{2in} \begin{verbatim} or \end{verbatim} \end{minipage} \\
  And & AND & \begin{minipage}{2in} \begin{verbatim} and \end{verbatim} \end{minipage} \\

  Equal (=) & EQUAL & \begin{minipage}{2in} \begin{verbatim} = \end{verbatim} \end{minipage} \\
  GreaterEqual ($\geq$) & GREATER\_EQUAL & \begin{minipage}{2in} \begin{verbatim} >= \end{verbatim} \end{minipage} \\
  Greater ($\textgreater$) & GREATER & \begin{minipage}{2in} \begin{verbatim} > \end{verbatim} \end{minipage} \\
  SmallerEqual ($\leq$) & SMALLER\_EQUAL & \begin{minipage}{2in} \begin{verbatim} <= \end{verbatim} \end{minipage} \\
  Smaller ($\textless$) & SMALLER & \begin{minipage}{2in} \begin{verbatim} < \end{verbatim} \end{minipage} \\
  Different ($\ne$) & DIFFERENT & \begin{minipage}{2in} \begin{verbatim} != \end{verbatim} \end{minipage} \\

  While & WHILE & \begin{minipage}{2in} \begin{verbatim} while \end{verbatim} \end{minipage} \\
  Do & DO & \begin{minipage}{2in} \begin{verbatim} do \end{verbatim} \end{minipage} \\
  Od & OD & \begin{minipage}{2in} \begin{verbatim} od \end{verbatim} \end{minipage} \\
  For & FOR & \begin{minipage}{2in} \begin{verbatim} for \end{verbatim} \end{minipage} \\
  From & FROM & \begin{minipage}{2in} \begin{verbatim} from \end{verbatim} \end{minipage} \\
  By & BY & \begin{minipage}{2in} \begin{verbatim} by \end{verbatim} \end{minipage} \\
  To & TO & \begin{minipage}{2in} \begin{verbatim} to \end{verbatim} \end{minipage} \\

  Print & PRINT & \begin{minipage}{2in} \begin{verbatim} print \end{verbatim} \end{minipage} \\
  Read & READ & \begin{minipage}{2in} \begin{verbatim} read \end{verbatim} \end{minipage} \\

  Number & NUMBER & \begin{minipage}{2in} \begin{verbatim} {[0-9]}* \end{verbatim} \end{minipage} \\
  VarName & VARNAME & \begin{minipage}{2in} \begin{verbatim} {{[A-Z]}|{[a-z]}}{{{[A-Z]}|{[a-z]}}|{[0-9]}}* \end{verbatim} \end{minipage} \\


\hline
\end{tabular}

\section{Implémentation}

\subsection{Mauvaise variable}
Pour éviter les cas où des noms de variables commencent par un numéro, un type \textit{BadVarName} a été implémenté pour retourner une erreur lorsque le cas se produit. 

\subsection{Compilation et exécution}
Pour lancer le projet, il faut d'abord lancer l'exécutable \textit{/more/compile.sh} qui se chargera de créer un .jar rendant l'exécution compatible avec la machine. 

\end{document}
